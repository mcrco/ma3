\documentclass[answers]{exam}
\makeindex

\usepackage{amsmath, amsfonts, amssymb, amstext, amscd, amsthm, makeidx, graphicx, hyperref, url, enumerate}
\newtheorem{theorem}{Theorem}
\allowdisplaybreaks

\begin{document}

\begin{center}
{\Large Ma 3 - Problem Set 5} \\
\medskip
Marco Yang \\
\medskip
2237027
\bigskip
\end{center}

RIP couldn't hit the buzzer beater gradescope submission so taking a 1 day extension on this.

\begin{questions}
\question [25] Find the MLE of $\theta$ when $X_1,X_2,\ldots,X_{n}$ are i.i.d.
random samples and each $X_{i}$ has a distribution given by the density function 
$f(x)=\frac{1}{2}e^{-|x-\theta|}$, $-\infty<x<\infty$.

\begin{solution}
The probability that every $X_{i}$ came from this distribution is


\[
P = \prod_{i=1}^{N} f(X_{i}) = \prod_{i=1}^{N} \frac{1}{2}e^{-|X_{i}-\theta|} = \frac{1}{2}e^{-\sum_{i=1}^{N}|X_{i}-\theta|}
.\] 

Maximizing this probability occurs at the critical point, e.g. when 
$\frac{\partial L}{\partial \theta} = 0$.

\begin{align*}
\frac{\partial L}{\partial \theta} &= \frac{1}{2}e^{-\sum_{i=1}^{N}|X_{i}-\theta|} \cdot \left( -\sum_{i=1}^{N} |X_{i} - \theta| \cdot \text{sign}(X_{i}-\theta) \right) \\
&= \frac{1}{2}e^{-\sum_{i=1}^{N}|X_{i}-\theta|} \cdot \left( -\sum_{i=1}^{N} (X_{i} - \theta) \right)  \\ 
&= -\frac{1}{2}e^{-\sum_{i=1}^{N}|X_{i}-\theta|} \left( \sum_{i=1}^{N} (X_{i} - \theta) \right)  \\ 
.\end{align*}

We know that $\sum_{i=1}^{N}(X_{i}-\theta) = 0$ when $\theta = \sum_{i=1}^{N}
X_{i} / N$. Thus, $\theta$ is the mean of the $X_{i}$.
\end{solution}

\question [25] The MSE of an estimator is given by $\text{MSE}(\hat{\theta}) 
= \mathbb{E}[(\hat{\theta}-\theta)^2]$. Show that 

\[
\text{MSE}(\hat{\theta}) = \text{Var}(\theta) + \text{Bias}(\theta)^2
.\] 

\begin{solution}
Note that $\hat{\theta}$ is independent of $\theta$. Thus, we can take it out of
any expected values w.r.t. $\theta$ as if it were a constant.

\begin{align*}
\text{MSE}(\hat{\theta}) &= \mathbb{E}[(\hat{\theta}-\theta)^2] \\ 
&= \mathbb{E}[\hat{\theta}^2 - 2\theta \hat{\theta} + \theta^2] \\ 
&= \mathbb{E}[\hat{\theta}^2] - 2\mathbb{E}[\theta \hat{\theta}] + \mathbb{E}[\theta^2] \\ 
&= \hat{\theta}^2 - 2\hat{\theta}\mathbb{E}[\theta] + \mathbb{E}[\theta^2] \\ 
&= \hat{\theta}^2 - 2\hat{\theta}\mathbb{E}[\theta] + \mathbb{E}[\theta]^2 + \mathbb{E}[\theta^2] - \mathbb{E}[\theta]^2 \\ 
&= (\mathbb{E}[\theta^2] - \mathbb{E}[\theta]^2) + (\hat{\theta} - \mathbb{E}[\theta])^2 \\ 
&= \text{Var}(\theta) + \text{Bias}(\theta)^2
.\end{align*}
\end{solution}

\question [25] Let $X_1,X_2,\ldots,X_{n}$ be an i.i.d. random sample where 
$X_{i}$ are distributed according to $U(0,\theta)$, i.e. a uniform distribution
on $[0,\theta]$ where $\theta$ is unknown.

\begin{parts}
\part Find $E(X)$ if $X \sim U(0,\theta)$.

\begin{solution}
\[
E(X) 
= \frac{1}{N}\sum_{i=1}^{N} \mathbb{E}[X_{i}] 
= \frac{1}{N} \sum_{i=1}^{N} \mathbb{E}[U(0, \theta)] 
= \frac{1}{N} \sum_{i=1}^{N} \frac{\theta}{2}
= \frac{\theta}{2}
.\] 
\end{solution}

\part Show that the estimator $\hat{\theta}=2\cdot \overline{x}$ is unbiased.

\begin{solution}
Plugging in $X_{i} \sim U(0, \hat{\theta}) = U(0, 2 \cdot \overline{x})$,

\[
E(X) 
= \frac{1}{N}\sum_{i=1}^{N} \mathbb{E}[X_{i}] 
= \frac{1}{N} \sum_{i=1}^{N} \mathbb{E}[U(0, 2 \cdot \overline{x})] 
= \frac{1}{N} \sum_{i=1}^{N} \overline{x}
= \overline{x}
.\] 

Since $E(X) = \overline{x}$ when $\hat{\theta} = 2 \cdot \overline{x}$,
the estimator is unbiased.
\end{solution}

\part Find the MLE for $\theta$.

\begin{solution}
The probability density for $U(0,\theta)$ is $\frac{1}{\theta}$. Thus, the
likelihood of getting our sample $X_{i}$ (or any valid sample) is

\[
p = \prod_{i=1}^{n} \frac{1}{\theta} = \theta^{-n}
.\] 

Taking the derivative w.r.t. $\theta$,

\[
\frac{dp}{dt} = -n\theta^{-n-1}
.\] 

Setting this to 0,

\[
\frac{dp}{dt} = -n\theta^{-n-1} = 0 \implies \theta = 0
.\] 

However, this doesn't make since since if $\theta=0$, then we know that all
values of $X$ must be 0, but we are constrained by the value we sampled. What we
can do is notice that $p$ is maximized for values of $\theta$ closer to 0. Thus,
we need to minimize $\theta$ given the sample, which would make 
$\theta = \text{max}(X_{i})$.
\end{solution}

\part Compare the MSE for the two estimators, which one is lower?

\begin{solution}
The MSE for $\hat{\theta} = 2 \cdot \overline{x}$ is

\[
\sum_{i=1}^{n} \left( X_{i} - \frac{2}{n} \sum_{i=1}^{n} X_{i} \right)^2
.\] 

Let $\text{max}(X_{i}) = m$. The MSE for $\theta = \text{max}(X_{i})$ is 

\[
\sum_{i=1}^{n} \left( X_{i} - m \right)^2
.\] 

Subtracting the first from the second,

\begin{align*}
\Delta &= \sum_{i=1}^{n} \left( 2X_{i} - \frac{2}{n} \sum_{i=1}^{n} X_{i} - m \right) \cdot \left( \frac{2}{n} \sum_{i=1}^{n} X_{i} + m \right)  \\ 
&= (2\overline{x} + m) \cdot \left( \left( 2\sum_{i=1}^{n} (X_{i} - \overline{x}) \right) - mn \right) \\ 
&= (2\overline{x} + m) (-mn ) \\ 
&= -mn(2\overline{x} + m) < 0
.\end{align*}

Thus, $\theta = \text{max}(X_{i})$ has a lower MSE.
\end{solution}
\end{parts}

\question[25] \textbf{(Hardy-Weinberg.)} Suppose that a particular gene occurs
as one of two alleles ($A$ and $a$), where allele $A$ has a frequency $\theta
\in (0,1)$ in the population. That is, a random copy of the gene is $A$ with
probability $\theta$ and $a$ with probability $1-\theta$. Since a diploid
genotype consists of two genes, the probability of each genotype is 

\[
P(AA) = \theta^2 \quad P(Aa) = 2\theta(1-\theta) \quad P(aa) = (1-\theta)^2
.\] 

Suppose we test a random sample of people and find that $n_1$ are $AA$, $n_2$
are $Aa$, and $n_3$ are $aa$. Find the MLE $\hat{\theta}_{n}$. Make sure to
verify that is indeed maximizing, by computing the second derivative of the
function you are maximizing.

\begin{solution}
The probability of getting the samples $n_{i}$ is

\[
p = \theta^2 \cdot 2\theta(1-\theta) \cdot (1-\theta)^2 = 2\theta^3(1-\theta)^3
= 2(\theta(1-\theta))^3
.\] 

To maximize this likelihood, we take the derivative w.r.t. $\theta$:

\[
\frac{dp}{d\theta} = 6(\theta(1-\theta))^2 \cdot (1 - 2\theta)
\] 

and find where it is zero:

\[
\frac{dp}{d\theta} = 0 \implies \theta = 0,1,\frac{1}{2}
.\] 

Out of these candidates, only $\theta = \frac{1}{2}$ makes sense since the other are 
the bounds for what a probability can be.

The second derivative is

\[
\frac{d^2p}{d\theta^2} = 6 \left( 2\theta(1-\theta)(1-2\theta)^2 - 2(\theta(1-\theta))^2 \right)
.\] 

At $\theta=\frac{1}{2}$, the second derivative is negative, meaning it is a
local maximum. At $\theta=0,1$, the second derivative is zero, which means we 
don't know if they are local minimums or maximums, but logically we can rule
them out because it would be impossible to get the given genotypes if we could
only have either $AA$ or $aa$.
\end{solution}
\end{questions}

\end{document}
