\documentclass[answers]{exam}
\makeindex

\usepackage{amsmath, amsfonts, amssymb, amstext, amscd, amsthm, makeidx, graphicx, hyperref, url, enumerate}
\newtheorem{theorem}{Theorem}
\allowdisplaybreaks

\begin{document}

\begin{center}
{\Large Ma 3 - Midterm} \\
\medskip
Marco Yang \\
\medskip
2237027
\bigskip
\end{center}

\begin{questions}
\question[10] A fair six-sided die is rolled once. Define the following events:

A: The result is even

B: The result is multiple of 3

C: The result is greater than 4

\begin{parts}
\part Find $P(A), P(B), and P(C)$.

\begin{solution}
\begin{align*}
    P(A) &= \frac{3}{6} = \frac{1}{2} \\ 
    P(B) &= \frac{2}{6} = \frac{1}{3} \\ 
    P(C) &= \frac{2}{6} = \frac{1}{3}
.\end{align*}
\end{solution}

\part Find $P(A \cup B)$, the probability of rolling either an even number or a
multiple of 3.

\begin{solution}
The numbers that satisfy either probability are $2,3,4,6$. Thus

\[
P(A \cup B) = \frac{4}{6} = \frac{2}{3}
.\] 
\end{solution}

\part Find $P(A \cap C)$, the probability of rolling a number that is both even
and greater than 4.

\begin{solution}
The only number that satisfies both conditions is 6. Thus,

\[
P(A \cap C) = \frac{1}{6}
.\] 
\end{solution}

\part Are events A and B mutually exclusive? Why?

\begin{solution}
No. There is an overlapping case with the number 6, which is both even and a
multiple of 3.
\end{solution}
\end{parts}

\question[5] A fair coin is tossed four times. What is the probability that the
number of heads appearing on the first two tosses is equal to the number of
heads appearing on the second two tosses?

\begin{solution}
Let $P(X)$ be the probability that the number of heads in 2 tosses is $X$. Since
we know the first two and second two tosses are independent, we have that the
probability of both the first two and second two tosses having $X$ heads is 

\[
\sum_{X}^{} P(X)^2 = \sum_{i=0}^{2} P(i)^2 = \left( \frac{1}{4} \right) ^2 + \left( \frac{1}{2} \right) ^2 + \left( \frac{1}{4} \right) ^2 = \frac{3}{8}
.\] 

We can also list out all permutations and see that $TTTT, THTH, HTHT, HTTH,
THHT, HHHH$ are the only 6 valid permutations out of the 16.
\end{solution}

\question[5] 5 fair dice are rolled. What is the probability that the faces
showing constitute a ``full'' house -- that is, three faces show one number and
two faces show a second number?

\begin{solution}
There are 6 ways of choosing the number that appears twice and 5 ways of
choosing the number that appears thrice. Thus, the total number of ways we can
pick the numbers for the full house is $5 \cdot 6 = 30$.

There are $\binom{5}{3} = 10$ ways of choosing which dice show the number that appear
thrice and which dice show the number that appear twice.

Lastly, the probability that each dice rolls the way we have designated them to
is $\frac{1}{6}^{5}$.

Thus, our probability of a full house is

\[
30 \cdot 10 \cdot \frac{1}{6^{5}} = 0.03858
.\] 
\end{solution}

\question[10] Prove that successive terms in the sequence $\binom{n}{0}, \ldots,
\binom{n}{n}$ first increase then decrease. For what value(j) of $j$ is 
$\binom{n}{j}$ maximized?

\begin{solution}
The ratio between successive terms is 

\[
r = \frac{\binom{n}{j+1}}{\binom{n}{j}} = \frac{n!}{(j+1)!(n - j - 1)!} \cdot \frac{j!(n-j)!}{n!} = \frac{n-j}{j+1}
.\] 

Thus, as long as $n-j > j-1$, the terms are increasing. Notice that after $n-j <
j-1$, the $\binom{n}{j}$ is decreasing. Since $j$ increases from $0$ to $n$
throughout the sequence, $n-j$ gets smaller while $j-1$ gets larger, and teh
sequence of binomial coefficientrs must be increasing then decreasing.

To find the maximizing values for $j$, we need to find the numbers for which
this pattern is non-decreasing, or $n-j \ge j-1$. Using this formula, we know
that $\binom{n}{j}$ is maximized when it is no longer non-decreasing, which is
$\frac{n}{2}$ for even $n$ and $\left\lfloor \frac{n}{2} \right\rfloor,
\left\lfloor \frac{n}{2} \right\rfloor+1$ for odd $n$.
\end{solution}

\question[10] A company ships products from three different warehouses (A, B,
and C). Based on customer complaints, it appears that 3\% of the shipments
coming from A are somehow faulty, as are 5\% of the shipments coming from B, and
2\% coming from C. Suppose a customer is mailed an order and calls in a
complaint the next day. What is the probability the item came from Warehouse C?
Assume that Warehouses A, B, and C ship 30\%, 20\%, and 50\% of the sales,
respectively.

\begin{solution}
We are looking for

\[
\frac{\text{Share of faults from C}}{\text{Share of faults}}
.\] 

Above, ``Share of $X$'' refers to the percentage share of items of type $X$ with
respect to the total inventory.

The share of faulty proudcts from $C$ is 

\[
0.02 \cdot 0.5 = 0.01
.\] 

The share of faulty products in general is

\[
0.03 \cdot 0.3 + 0.05 \cdot 0.2 + 0.02 \cdot 0.5 = 0.11
.\] 

Thus, the probability that the item came from Warehouse C is $\frac{1}{11}$.
\end{solution}

\question[15] Let $X$ be an exponentai r.v. with standard deviation $\sigma$.
Find $P(|XE(X)|>k\sigma)$ for $k=2,3,4$ and compare the results to the bounds
from Chebyshev's inequality.

\begin{solution}
For exponential r.v.s,

\begin{align*}
    E(X) &= \frac{1}{\lambda} = \sigma \\
    Var(X) &= \frac{1}{\lambda^2} = \sigma^2
.\end{align*}

We know that the standard deviation is the square root of the variance, which is
the square of the expected value. Thus, the standard deviation is the expected
value. Since we know $X$ only takes on positive values,

\[
P(|XE(X)| > k\sigma) = P(X\sigma > k\sigma) = P(X > k) = 1 - P(X \le k) = 1 - (1 - e^{-\lambda k})
= e^{-\frac{k}{\sigma}}
.\] 

Because this has the $\sigma$ term in it still, I'm assuming the problem meant
to ask $P(|X - E(X)| > k\sigma)$? If so,

\begin{align*}
P(|X-E(X)| > k\sigma) &= P(X-\sigma > k\sigma) \\
&= P(X > (k+1)\sigma)  \\ 
&= 1 - P(X \le (k+1)\sigma) \\ 
&= 1 - (1 - e^{-\lambda (k+1)\sigma}) \\
&= e^{-(k+1)}
.\end{align*}

Chebyshev's Inequality states

\[
P(|X - E(X)| \ge a) \le \frac{\sigma^2}{a^2}
.\] 

Thus, if $a = k\sigma$,

\[
P(|X - E(X)| \ge k\sigma) \le \frac{\sigma^2}{k^2\sigma^2} = \frac{1}{k^2}
.\] 

Now, for $k=2,3,4$,

\begin{align*}
P(|X - E(X)| > 2\sigma) &= e^{-(2+1)} = 0.04979 \le \frac{1}{4} = 0.25 \\ 
P(|X - E(X)| > 3\sigma) &= e^{-(3+1)} = 0.01832 \le \frac{1}{9} = 0.1111 \\ 
P(|X - E(X)| > 2\sigma) &= e^{-(4+1)} = 0.006738 \le \frac{1}{16} = 0.00675 \\ 
.\end{align*}

It's clear that as $k$ increases, the Chebyshev bound gets sharper rapidly.
\end{solution}

\question[10] Suppose $X$ is a binomial r.v. with $n=4$ and $p=2 / 3$. What is
the pdf of $2X + 1$?

\begin{solution}
Let $Y = 2X + 1$. $Y$ must be odd and is limited by the support of $X$, so we limit 
our support to only odd integers from $1$ through $2n + 1$. Let $P$ be the pdf
of $Y$ and $Q$ be the binompdf for $X$. We know 

\[
P(Y = y) = Q(X = \frac{y-1}{2})
.\] 

Plugging in the binompdf formula for $n=4$ and $p=2 / 3$,  

\[
P(Y = y) = \begin{cases}
(\frac{2}{3})^{\frac{y-1}{2}}\left( \frac{1}{3} \right)^{4-\frac{y-1}{2}}
\binom{4}{\frac{y-1}{2}} &\quad \text{if } y \in \{1,3,5,7,9\} \\ 
0 &\quad \text{otherwise}
\end{cases}
.\] 
\end{solution}

\question[10] Find the pdf for the discrete r.v. $X$ whose cdf at the points
$x=0,1,\ldots,6$ are given by $F_{X}(x) = x^3 / 216$.

\begin{solution}
We know 

\[
P(x) = F_{X}(x) - F_{X}(x - 1)
.\] 

Thus,

\begin{align*}
    P(0) &= F_{X}(0) = 0 \\ 
    P(1) &= F_{X}(1) - F_{X}(0) = \frac{1}{216} \\
    P(2) &= F_{X}(2) - F_{X}(1) = \frac{7}{216} \\
    P(3) &= F_{X}(3) - F_{X}(2) = \frac{19}{216} \\
    P(4) &= F_{X}(4) - F_{X}(3) = \frac{37}{216} \\
    P(5) &= F_{X}(5) - F_{X}(4) = \frac{61}{216} \\
    P(6) &= F_{X}(6) - F_{X}(5) = \frac{91}{216} \\ 
    P(x) &= 0 \quad \text{for } x \not\in \{1,2,3,4,5,6\}
.\end{align*}
\end{solution}

\question[15] Suppose that $X$ has the pdf $f(x)=cx^2$ for $0\le x\le 1$ and
$f(x) = 0$ otherwise.

\begin{parts}
\part Find $c$.

\begin{solution}
We know the corresponding cdf must sum to 1.

\[
\int_{0}^{1} cx^2 \, dx = 1 \implies c \cdot \frac{1}{3} = 1 \implies c = 3
.\] 
\end{solution}

\part Find the cdf.

\begin{solution}
\[
F(t) = \int_{0}^{t} f(x) \, dx = \int_{0}^{t} 3x^2 \, dx = t^3
.\] 
\end{solution}

\part What is $P(0.1 \le X \le 0.5)$?

\begin{solution}
\[
\int_{0.1}^{0.5} f(x) \, dx = 0.5^3 - 0.1^3 = 0.124
.\] 
\end{solution}
\end{parts}

\question[20] Let $X$ be a r.v. with the cdf $F(x) = 1 - x^{-\alpha}, x\ge 1$.

\begin{parts}
\part For what values of $\alpha$ is this a valid cdf?

\begin{solution}
Valid cdfs are increasing, right-continuous, and converge to $0$ and $1$ in the
limits.

Since $x\ge 1$ and pdf $f(x) = \frac{d}{dx}F(x) = \alpha x^{-\alpha-1} > 0 \implies
\alpha > 0$, we know $\alpha > 0$.

Since $1$ (a constant) and $-x^{-\alpha}$ are both continuous for all $\alpha$,
the right-continuity condition doesn't set any bounds.

$F(1) = 1 - 1^{-\alpha} = 0$ is true for all $\alpha$. $\lim_{x \to \infty} 
F(x) = 1 - x^{-\alpha} = 1 \implies \alpha > 0$.

Thus, we have that $\alpha > 0$.
\end{solution}

\part For what values of $\alpha$ is $E(X)$ finite? Find $E(X)$ for those values
of $\alpha$ for which $E(X)$ exists.

\begin{solution}
\begin{align*}
E(X) &= \int_{1}^{\infty} xf(x) \, dx  \\
&= \int_{1}^{\infty} x\alpha x^{-\alpha-1} \, dx  \\
&= \int_{1}^{\infty} \alpha x^{-\alpha} \, dx \\
&= \lim_{x \to \infty} \frac{\alpha}{-\alpha + 1} x^{-\alpha + 1} - \frac{\alpha}{-\alpha + 1} 1^{-\alpha + 1} \\
&= \lim_{x \to \infty} \frac{\alpha}{-\alpha + 1} x^{-\alpha + 1} - \frac{\alpha}{-\alpha + 1}
.\end{align*}

For $E(X)$ to be finite, we need $x^{-\alpha+1}$ to be finite as $x$ approaches
0 and $-\alpha + 1 \neq 0$. Thus, $\alpha > 1$, and

\[
    E(X) = \frac{\alpha}{\alpha - 1}
.\] 
\end{solution}

\part For what values of $\alpha$ is $Var(X)$ finite? Find $Var(X)$ for those
values of $\alpha$ for which it exists.

\begin{solution}
\begin{align*}
Var(X) &= E(X^2) - (E(X))^2 \\
E(X^2) &= \int_{1}^{\infty} x^2f(x) \, dx  \\
&= \int_{1}^{\infty} x^2\alpha x^{-\alpha-1} \, dx  \\
&= \int_{1}^{\infty} \alpha x^{-\alpha + 1} \, dx \\
&= \lim_{x \to \infty} \frac{\alpha}{-\alpha + 2} x^{-\alpha + 2} - \frac{\alpha}{-\alpha + 2} 1^{-\alpha + 2} \\
&= \lim_{x \to \infty} \frac{\alpha}{-\alpha + 2} x^{-\alpha + 2} - \frac{\alpha}{-\alpha + 2}
.\end{align*}

For $Var(X)$ to be finite, we need $E(X^2)$ and $(E(X))^2$ to both be finite.
For the latter, we have the same condition from part b, which is $\alpha > 1$.
For the former, we need $x^{-\alpha+2}$ to be finite as $x$ approaches
0 and $-\alpha + 2 \neq 0$. Thus, $\alpha > 2$, and

\[
    Var(X) = E(X^2) - (E(X))^2 = \frac{\alpha}{\alpha - 2} - \frac{\alpha^2}{\alpha^2 - 2\alpha + 1}
.\] 
\end{solution}

\part Are the values of $\alpha$ the same for all parts above?

\begin{solution}
No?
\end{solution}
\end{parts}
\end{questions}
\end{document}
