\documentclass[answers]{exam}
\makeindex

\usepackage{amsmath, amsfonts, amssymb, amstext, amscd, amsthm, makeidx, graphicx, hyperref, url, enumerate}
\newtheorem{theorem}{Theorem}
\allowdisplaybreaks

\begin{document}

\begin{center}
{\Large Ma 3 - Problem Set 1} \\
\medskip
Marco Yang \\
\medskip
2237027
\bigskip
\end{center}

\begin{questions}
\question [20] In bridge, there are 4 players (A, B, C, D) and each player
receives 13 cards from standard shuffled 52 card deck. What is the probability
that

\begin{parts}
\part exactly 1 of 4 players has one ace and one king?

\begin{solution}
The probability that exactly 1 of the 4 players has one ace and one king
is the ratio of the number of hands such that exactly 1 player has one ace
and one king and the rest do not. Let a hand with one ace and one king be
denoted as a valid hand.

The number of ways we can choose one player to have a valid hand is

\[
    A = 4 \cdot 4^2 \cdot \binom{44}{11}
\] 

since there are 4 players from which we can choose one to have a valid hand,
4 aces and 4 kings we can choose from to form a pair, and there are 44 
remaining non-king and non-ace cards from which we can choose 11 cards 
to create the rest of the hand.

Now, we want the ways we can form the remaining hands such that no player
has a king and an ace. This is the same as finding the complement of the 
number of permutations of hands such that there is at least one valid hand.
Notice that to count the complement, we only have to consider two cases:

\begin{enumerate}
    \item exactly 1 of the remaining 3 players has a valid hand
    \item exactly 3 of the remaining 3 players has a valid hand (there is no
        way for exactly 2 of the remaining 3 players to have a valid hand
        because the last king and last ace that isn't in the hands of the 
        first three players must go to the last player, in which case the last
        player also has a valid hand)
\end{enumerate}

For the first case, the number ways we can choose exactly 1 remaining player 
to have a valid hand is

\[
3 \cdot 3^2 \cdot \binom{33}{11} \cdot X
\] 

since there 3 remaining players from which we can choose one to have a valid
hand, $3^2$ ways to choose an ace-king pair from the remaining cards, 33 
remaining non-ace and non-king cards from which we choose 11 to complete the 
hand, and $X$ is the number of ways we can choose the remaining two hands 
such that neither hand is valid.

Notice that $X$ is the complement of the number of ways we can choose the
hands such that both of the remaining players have valid hands (we can't have
exactly 1 out of the 2 remaining players to have a valid hand since that would
cause the other player to also have a valid hand, as explained before). Thus
the complement of $X$, which we denote $X^{c}$, is

\[
X^{c} = 2^2 \cdot \binom{22}{11}
\] 

since there are $2^2$ ways to choose an ace-king pair and 22 remaining 
non-king and non-ace cards from which we choose 11 to complete the players 
hand. We need not compute the number of ways to form the last players hand 
since it's already predetermined by the other 3 players' hands.

Thus, for case 1, the number of ways we can choose exactly 1 player out of 3 
remaining players to have a valid hand is

\[
3 \cdot 3^2 \cdot \binom{33}{11} \cdot (1 - X^{c}) = 3 \cdot 3^2 \cdot \binom{33}{11} \cdot \left( \binom{26}{13} - 2^2 \cdot \binom{22}{11} \right) 
.\] 

For case 2, the number of ways we can we choose all 3 remaining players to have
a valid hand is 

\[
3^2 \cdot \binom{33}{11} \cdot X^{c}
\] 

since there are $3^2 \cdot \binom{33}{11}$ ways for the first remaining player 
to have a valid hand, and $X^{c}$ ways to have the last 2 remaining players to 
also have valid hands. Thus, our final number of ways to have at least 1 out of
the 3 remaining players to have valid hands is

\[
Y = 3 \cdot 3^2 \cdot \binom{33}{11} \cdot \left( \binom{26}{13} - 2^2 \cdot \binom{22}{11} \right) + 3^2 \cdot \binom{33}{11} \cdot 2^2 \cdot \binom{22}{11}
.\] 

Back to the original problem: we are looking to multiply the number of ways to
have one player have a valid hand and the rest to have invalid hands. Thus, we
are looking for the complement of the above number, which is 

\[
B = \binom{39}{13} \cdot \binom{26}{13} \cdot \binom{13}{13} - Y
.\] 

The total number of permutations of the four player's hands is

\[
\binom{52}{13} \cdot \binom{39}{13} \cdot \binom{26}{13} \cdot \binom{13}{13}
\] 

Putting it all together, the probability that exactly one player has a 
valid hand is

\[
\frac{A \cdot B}{\binom{52}{13} \cdot \binom{39}{13} \cdot \binom{26}{13} \cdot \binom{13}{13}} = 0.3656
.\] 
\end{solution}

\part player A and B each have exactly one ace?

\begin{solution}
There are $4 \cdot 3$ permutations for the ace that A and B receive, and 
the probability that player A's ace goes to player A is $\frac{13}{52}$,
the probablity that player B's ace goes to player B is $\frac{13}{51}$ since
the first ace removes a potential spot for player B to go to, and the same
approach is used to find the probabilities that the remaining two aces go to
either player C or D.

\[
4 \cdot 3 \cdot \left( \frac{13}{52} \right) 
\cdot \left( \frac{13}{51} \right) \cdot \left( \frac{26}{50} \right) 
\cdot \left( \frac{25}{49} \right) = 0.2029
.\] 
\end{solution}
\end{parts}

\question [20] A club consists of 10 seniors, 12 juniors, and 15 sophomores.
An organizing committee of size 5 is chosen randomly (with all subsets of size
5 equally likely).

\begin{parts}
\part Find the probability that there are exactly 3 sophomores in the comittee.

\begin{solution}
There are $\binom{15}{3}$ combinations of juniors and $\binom{22}{2}$ combos
of seniors/juniors. The total number of combinations is $\binom{37}{5}$. 
Thus, the probability is

\[
P = \frac{\binom{15}{3}\binom{22}{2}}{\binom{37}{5}} = 0.2411
.\] 
\end{solution}

\part Find the probability that the committee has at least one rep
from each of the senior, junior, and sophomore classes.

\begin{solution}
I think the complement is easier to calculate. The probability that the 
the committee is formed by at most 2 classes is the combinations of 5
people chosen from only seniors/juniors, only juniors/sophs, and only
seniors/sophs, minus the number of committees only formed by one class 
since they overlap. 

Thus,

\begin{align*}
    P &= 1 - \left(\frac{\binom{10 + 12}{5} + \binom{12 + 15}{5} + 
    \binom{10 + 15}{5} - (\binom{10}{5} + \binom{12}{5} + 
    \binom{15}{5})}{\binom{37}{5}}\right) \\ 
    &= 0.6418
.\end{align*}
\end{solution}
\end{parts}

\question [15] Let $A$ and $B$ be events. The symmetric difference $A\Delta B$ 
is defined to be the set of all elements that are in $A$ and $B$ but not both.
In logic and engineering this event is alos called the XOR of $A$ and $B$. 
Show that 

\[
P(A \Delta B) = P(A) + P(B) - 2P(A \cap B)
.\] 

\begin{solution}
\begin{proof}
By PIE, we know that 

\[
P(A \cup B) = P(A) + P(B) - P(A \cap B)
.\] 

Since $A \Delta B = (A \cup B) \setminus (A \cap B)$,

\[
P(A \Delta B) = P(A \cup B) - P(A \cap B) = P(A) + P(B) - 2P(A \cap B)
.\] 
\end{proof}
\end{solution}

\question [10] Suppose $A$ and $B$ are independent events. Show that $A^{c}$
and $B^{c}$ are also independent.

\begin{solution}
\begin{proof}
From DeMorgan's Law, we know that the intersection of the complements of 
two sets is the complement of the union. Thus, we have

\begin{align*}
P(A^{c} \cap B^{c}) &= P((A \cup B)^{c}) \\ 
&= 1 - P(A \cup B) \\ 
&= 1 - (P(A) + P(B) - P(A \cap B)) \\
&= 1 - P(A) - P(B) + P(A \cap B)
.\end{align*}

Since $A$ and $B$ are independent, we know that $P(A \cap B) = P(A)P(B)$.
Thus,

\begin{align*}
P(A^{c} \cap B^{c}) &= 1 - P(A) - P(B) + P(A)P(B) \\ 
&= (1 - P(A))(1 - P(B))
.\end{align*}

By definition, $P(A^{c}) = 1 - P(A)$ and $P(B^{c}) = 1 - P(B)$. Thus,

\[
P(A^{c} \cap B^{c}) = P(A^{c})P(B^{c})
.\] 

Thus, $A^{c}$ and $B^{c}$ fulfill the criteria for independence.
\end{proof}
\end{solution}

\question [10] Two cards are distributed to each of three players from a 
standard deck of 52 shuffled cards. What is the probability that at least two 
of the three players have an ace with either a jack or a king?

\begin{solution}
There are two cases: exactly two of the three players have an ace with either 
a jack or a king or exactly three players have an ace with either a jack or a 
king.

For the first case, there are $\binom{3}{2} = 3$ to choose the two players 
who have an ace with either a jack or a king and the player who doesn't have 
an ace with either a jack or a king. The probability that both selected 
players satisfy the conditions is

\[
\frac{4 \cdot 8}{\binom{52}{2}} \cdot \frac{3 \cdot 7}{\binom{50}{2}}
.\] 

The probability that the third player does not have an ace with either
a king or a jack is the complement of the probability that the third
player does have an ace with a king or a jack, which is

\[
1 - \frac{2 \cdot 6}{\binom{48}{2}}
.\] 

Thus, the probability for the first case is

\[
P_1 = 3 \cdot \frac{4 \cdot 8}{\binom{52}{2}} \cdot \frac{3 \cdot 7}{\binom{50}{2}} \cdot \left( 1 - \frac{2 \cdot 6}{\binom{48}{2}} \right) 
.\] 

For the second case, there is $\binom{3}{3} = 1$ way to choose which players
have an ace with either a jack or a king. The probability that all three have
an ace with either a jack or a king is

\[
P_2 = \frac{4 \cdot 8}{\binom{52}{2}} \cdot \frac{3 \cdot 7}{\binom{50}{2}} \cdot \frac{2 \cdot 6}{\binom{48}{2}}
.\] 

Thus, our answer is

\[
P_1 + P_2 = 0.001232
.\] 
\end{solution}

\question [25] Alice and Bob play a game. Alice has two blue chips, while Bob 
has three red chips. They put all five chips into a bag, mix them up, then an 
impartial referee removces them one at a time and returns them to the players.
The winner is the first person to collect all their chips.

\begin{parts}
\part Determine the probability that Alice wins this game, and explain how you
calculated your answer.

\begin{solution}
The probability that Alice wins is the probability of a permutation of 
5 chips with the last chip being red, which is just

\[
    \frac{\binom{4}{2}}{\binom{5}{2}} = \frac{3}{5}
.\] 
\end{solution}
\part In the next round, Alice has ony one blue chip, while Bob has six
red chips. Now what is the probability that Alice wins, and why?

\begin{solution}
The probability that Alice wins is the probability of a permutation of 
5 chips with the last chip being red, which is just the probability that 

\[
    \frac{\binom{6}{1}}{\binom{7}{1}} = \frac{6}{7}
.\] 
\end{solution}

\part Make a conjecture for different numbers of red and blue chips.

\begin{solution}
Notice that our set of all permutations of the chips is the same as the
set of all backwards permutations of the chips. Thus, WLOG, we can let
the last chip be the first one we "choose" from the set of all chips. 
Since the probability of Alice winning is the same as the last chip
being red, it's just the probability that the first chip we choose from the
bag is red, or the ratio of the red chips to all chips:

\[
\frac{\text{number of red chips}}{\text{number of total chips}}
.\] 
\end{solution}
\end{parts}
\end{questions}

\end{document} 
