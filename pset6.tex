\documentclass[answers]{exam}
\makeindex

\usepackage{amsmath, amsfonts, amssymb, amstext, amscd, amsthm, makeidx, graphicx, hyperref, url, enumerate}
\newtheorem{theorem}{Theorem}
\allowdisplaybreaks

\begin{document}

\begin{center}
{\Large Ma 3 - Problem Set 5} \\
\medskip
Marco Yang \\
\medskip
2237027
\bigskip
\end{center}

\begin{questions}
\question [25] A coin is thrown independently 10 times to test the hypothesis
that the probability of heads is $p=0.5$ versus the alternative that the
probability is not 0.5. The test rejects if either 0 or 10 heads are observed.

\begin{parts}
\part What is the significance level $(\alpha)$ of the test?

\begin{solution}
The significance, which is the probability that the hypothesis $p=0.5$ is
rejected when it is true, is $\alpha = 2 \cdot 0.5^{10} = 0.5^{9}$. 
\end{solution}

\part If, in fact, the probability of heads is 0.10, what is the power of the
test?

\begin{solution}
The probability of rejecting $p=0.5$ (e.g. we observe 10 heads or 0 heads) when
$p=0.1$ is $1-\beta = 0.1^{10} + (1-0.1)^{10} \implies \beta = 1 - (0.1^{10} +
0.9^{10})$.
\end{solution}
\end{parts}

\question [25] A producer specifies that the mean lifetime of a certain type of
battery is at least 240 hours. A sample of 18 such batteries yields the
following data:

\[
237, 242, 244, 262, 225, 218, 242, 248, 243, 234, 236, 228, 232, 230, 254, 220, 232, 240
.\] 

Assuming that the life of the batteries is approximately normally distributed,
do the data indicate that the specifications are not being met?

\begin{solution}
The mean of the data is $\overline{X} = 237.0556$ and the standard deviation is 
$s = 11.2797$. The $t$ score is 

\[
t = \frac{\overline{X} - \mu_0}{s / \sqrt{n}} = \frac{237.0556 - 240}{11.2797 /
\sqrt{28}} = -1.1075
.\] 

The probability of the battery life distribution having a true mean of at least
240 is $0.1418$, which is not insignificant. Thus, we can't conclude that the
batteries have less than 240 hours of battery life.
\end{solution}

\question [25] Consider two different producers of a type of rocket powder. A
sample of 10 units is selected from each producer and their burning times in
seconds are recorded. This results in the data below. Test the hypothesis that
the average burning times are equal. What assumptions are you making?

Producer 1

\[
50.7, 54.8, 48.6, 36.9, 52.4, 51.6, 53.0, 38.0, 42.2, 50.3
.\] 

Producer 2

\[
60.3, 58.8, 56.2, 48.6, 40.0, 42.8, 58.0, 44.3, 55.0, 48.6
.\] 

\begin{solution}
I'm going to test this with the two sample t-test. We are assuming that they are
independent and both normally distributed with the same variance.

The distribution attributes are 

\begin{align*}
    \overline{X}_{1} &= 47.85 \\ 
    s_1 = 6.4388 \\
    n_1 &= 10 \\
    \overline{X}_{2} &= 51.26 \\ 
    s_2 = 7.3284 \\
    n_2 &= 10 \\ 
    s_{p} = 7.1282
.\end{align*}

The $t$ statistic is

\[
t = \frac{\overline{X}_{1} - \overline{X}_{2}}{s_{p}\sqrt{\frac{1}{n_1} + \frac{1}{n_2}}} 
= \frac{47.85 - 51.26}{7.1282 \sqrt{\frac{1}{10} + \frac{1}{10}}} 
= -1.0697
.\] 

The probability of $|t| > 1.0697$ with $df=10 + 10 - 2 = 18$ is $0.298$. Since this
probability is not insignificant (less than 0.05), we can't say that they are
equal.
\end{solution}

\question [25] Supose that $X_1,X_2,\ldots,X_{n}$ is a random sample from a
$N(\mu, \sigma^2)$ distribution.

\begin{parts}
\part If $\sigma^2$ is known, find a minimum value for $n$ to guarantee that a
0.95 confidence interval for $\mu$ will have length no more than $\sigma / 4$.

\begin{solution}
The length for a 0.95 confidence interval for a sample size of 1 is $3.92 \sigma$.
Since the interval length is proportional to the inverse square root of $n$, we
want

\[
    \frac{3.92\sigma}{\sqrt{n}} \le \frac{\sigma}{4} \implies n \ge 246
.\] 
\end{solution}

\part If $\sigma^2$ is unknown, find a minimum value for $n$ to guarantee, with
probability 0.90, that a 0.95 confidence interval for $\mu$ will have length
no more than $\sigma / 4$.

\begin{solution}
We know that the ratio between the square of the sample variance and the square
of the distribution variance follows the chi squared distribution. Thus, we can
use the chi-squared distribution with $n-1$ degrees of freedom to find the range
of $s$ with probability 0.9. That is, for

\[
\chi^2 = \frac{(n-1)s^2}{\sigma^2} \le \chi_{0.9}^2
,\] 

where $P(\chi^2 \le \chi_{0.9}^2) = 0.9$ for a chi-squared distribution with
$n-1$ degrees of freedom. Rearranging, we have

\[
s \le \sigma \sqrt{\frac{\chi_{0.9}^2}{n-1}}
,\] 

Then, approximating the distribution for $X$ with a $t$ distribution, we can
find $n$ based on the constraints of the confidence interval using the above
formula for $s$ and $t$ score corresponding to 0.95 confidence with $n-1$
degrees of freedom, which we denote $t_{0.95}$:

\[
\frac{2t_{0.95}s}{\sqrt{n}} \le \frac{\sigma}{4} \implies \frac{2t_{0.95}}{\sqrt{n}} \cdot \sigma \sqrt{\frac{\chi_{0.9}^2}{n-1}} \le \frac{\sigma}{4} \implies n(n-1) \ge (8t_{0.95}\chi_{0.9}^2)
.\] 

Then, we just have to check this with precomputed $\chi^2$ and $t$ score
tables, but I'm a lazy bum so I'm not gonna do it.
\end{solution}
\end{parts}
\end{questions}
\end{document}
